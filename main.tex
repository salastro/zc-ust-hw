\documentclass{zc-ust-hw}

\usepackage{lipsum}

\newcommand*{\name}{SalahDin Rezk}
\newcommand*{\id}{202201079}
\newcommand*{\course}{Introduction to Classical Mechanics (PHYS 101)}
\newcommand*{\assignment}{Assignment 1}

\begin{document}

\maketitle

\begin{enumerate}

    %%%%%%%%%%%%%%%%%%%%
    \item Question Number 1
        %%%%%%%%%%%%%%%%%%%%

        \begin{enumerate}
            \item Escriba un programa en \verb|python| para calcular un valor aproximado de la integral
                \begin{equation}
                    \int_0^2\left(x^4 - 2x +1 \right)dx
                \end{equation}
            \item Escriba un programa que calcule la integral usando la regla de Simpson con 10 \textit{slices}.
            \item Compare los resultados con el valor exacto (integre). ¿Cuál es el error fraccional de sus cálculos?
            \item Modifique el programa para utilizar cientos de \textit{slices}, y luego miles. Note la mejora en el resultado. Compare estos resultados con la regla del trapecio utilizando este gran numero de \textit{slices}.
        \end{enumerate}

        %%%%%%%%%%%%%%%%%%%%
    \item Question Number 2
        %%%%%%%%%%%%%%%%%%%%

        \lipsum[2]

        \noindent

        \begin{equation}
            J_m(x)=\frac{1}{\pi}\int_0^\pi \cos(m\theta-x\sin(\theta))d\theta
        \end{equation}

        \noindent
        \lipsum[1]

        Solve

        \begin{enumerate}
            \item \lipsum[66]
            \item \lipsum[75]
        \end{enumerate}

        %%%%%%%%%%%%%%%%%%%%
    \item Question Number 3
        %%%%%%%%%%%%%%%%%%%%

        \lipsum[23]
        \begin{itemize}
            \item \lipsum[75]
            \item \lipsum[66]
        \end{itemize}

\end{enumerate}

\end{document}
